\documentclass[12pt]{exam}


% Symbols
\usepackage{amsmath, amsthm, amssymb}
\usepackage{mathtools}
\usepackage{wasysym}
\usepackage{mathrsfs}  
\usepackage{stmaryrd}
\usepackage{comment}
\usepackage{marginnote}

% Style
\usepackage[margin=1in]{geometry}
\usepackage{parskip}
\usepackage[shortlabels]{enumitem}

% Colored text
%\usepackage{color}
\usepackage{xcolor}

% Links
\usepackage[bookmarks,colorlinks,breaklinks]{hyperref}
% PDF hyperlinks, with coloured links
\hypersetup{linkcolor=red,citecolor=blue,
  filecolor=dullmagenta,urlcolor=blue}

% Bibliography and references
\usepackage[numbers]{natbib}
\usepackage[capitalize]{cleveref}

% Figures
\usepackage{graphicx}
\usepackage{tikz}
\usetikzlibrary{matrix, arrows}
\usetikzlibrary{cd}

\NeedsTeXFormat{LaTeX2e}
\ProvidesPackage{quiver}[2021/01/11 quiver]

% `tikz-cd` is necessary to draw commutative diagrams.
\RequirePackage{tikz-cd}
% `amssymb` is necessary for `\lrcorner` and `\ulcorner`.
\RequirePackage{amssymb}
% `calc` is necessary to draw curved arrows.
\usetikzlibrary{calc}
% `pathmorphing` is necessary to draw squiggly arrows.
\usetikzlibrary{decorations.pathmorphing}

% A TikZ style for curved arrows of a fixed height, due to AndréC.
\tikzset{curve/.style={settings={#1},to path={(\tikztostart)
    .. controls ($(\tikztostart)!\pv{pos}!(\tikztotarget)!\pv{height}!270:(\tikztotarget)$)
    and ($(\tikztostart)!1-\pv{pos}!(\tikztotarget)!\pv{height}!270:(\tikztotarget)$)
    .. (\tikztotarget)\tikztonodes}},
    settings/.code={\tikzset{quiver/.cd,#1}
        \def\pv##1{\pgfkeysvalueof{/tikz/quiver/##1}}},
    quiver/.cd,pos/.initial=0.35,height/.initial=0}
\usepackage[all]{xy}
   \SelectTips{cm}{10}

\usepackage{hyperref}
\usepackage{enumitem}
\usepackage{graphicx}

\makeatletter
\def\thm@space@setup{%
  \thm@preskip=0.5em\thm@postskip=\thm@preskip%
}
\makeatother
% Unnamed theorem without counter
\newtheorem*{theorem}{Theorem}
\newtheoremstyle{named}{}{}{\\itshape}{}{\bfseries}{.}{.5em}{\thmnote{#3's }#1}
% Named theorem without counter
\theoremstyle{named}
\newtheorem*{namedtheorem}{Theorem}
% Other theorem environments
\theoremstyle{plain}
\newtheorem{thm}{Theorem}[section]
\newtheorem{conj}[thm]{Conjecture}
\newtheorem{prop}[thm]{Proposition}
\newtheorem{lem}[thm]{Lemma}
\newtheorem{sublem}[thm]{Sub-Lemma}
\newtheorem{cor}[thm]{Corollary}
\theoremstyle{definition}
\newtheorem{defn}[thm]{Definition}
\newtheorem{ex}[thm]{Exercise}
\newtheorem{eg}[thm]{Example}
\newtheorem{assumption}[thm]{Assumption}
\newtheorem{convention}[thm]{Convention}
\newtheorem{fact}{Fact}
\crefformat{footnote}{#2\footnotemark[#1]#3}

%Hom
\newcommand{\Hom}{\mathrm{Hom}}

% Fonts and notation
\newcommand{\CC}{\mathbb{C}}
\newcommand{\QQ}{\mathbb{Q}}
\newcommand{\RR}{\mathbb{R}}
\newcommand{\ZZ}{\mathbb{Z}}

\newcommand{\mf}[1]{\mathfrak{#1}}
\newcommand{\mc}[1]{\mathcal{#1}}
\newcommand{\ms}[1]{\mathscr{#1}}
\newcommand{\mr}[1]{\mathrm{#1}}
\newcommand{\mbb}[1]{\mathbb{#1}}
\newcommand{\ol}[1]{\overline{#1}}
\newcommand{\ul}[1]{\underline{#1}}
\newcommand{\wt}[1]{\widetilde{#1}}
\newcommand{\wh}[1]{\widehat{#1}}

% Geometry
\newcommand{\Spa}{\mathrm{Spa}}
\newcommand{\Spec}{\mathrm{Spec}}
\newcommand{\Spf}{\mathrm{Spf}}
\newcommand{\Spd}{\mathrm{Spd}}

\newcommand{\an}{\mathrm{an}}

% sites
\newcommand{\proet}{\mr{pro\acute{e}t}}
\newcommand{\et}{\mr{\acute{e}t}}
\newcommand{\fet}{\mr{f\acute{e}t}}


%Algebraic groups
\newcommand{\GL}{\mathrm{GL}}
\newcommand{\SL}{\mathrm{SL}}
\newcommand{\SO}{\mathrm{SO}}
\newcommand{\OO}{\mathrm{O}}
\newcommand{\Sp}{\mathrm{Sp}}
\newcommand{\GSp}{\mathrm{GSp}}
\newcommand{\Spin}{\mathrm{Spin}}
\newcommand{\GSpin}{\mathrm{GSpin}}
\newcommand{\SU}{\mathrm{SU}}
\newcommand{\Stab}{\mathrm{Stab}}
\newcommand{\Rep}{\mathrm{Rep}}

% Miscellaneous
\newcommand{\Frob}{\mathrm{Frob}}
\DeclareMathOperator{\Aut}{Aut}
\newcommand{\colim}{\mathrm{colim}}
\renewcommand{\l}{\left}
\renewcommand{\r}{\right}

% p-adic geom
\newcommand{\crys}{\mathrm{crys}}
\newcommand{\Fil}{F}
\newcommand{\HT}{\mathrm{HT}}
\newcommand{\Gr}{\mathrm{Gr}}
\newcommand{\Fl}{\mathrm{Fl}}
\newcommand{\cris}{\mathrm{cris}}
\newcommand{\Isoc}{\mr{Isoc}} 
\newcommand{\dR}{\mathrm{dR}}
\newcommand{\Perf}{\mathrm{Perf}}
\newcommand{\Perfd}{\mathrm{Perfd}}
\newcommand{\FF}{\mathrm{FF}}


% Shimura varieties
\newcommand{\Sh}{\mathrm{Sh}}

% Coefficient objects for Weil cohomology theories
\newcommand{\Loc}{\mathrm{Loc}}
\newcommand{\MIC}{\mathrm{MIC}}

%Spacing
\newcommand{\+}{\, }


\begin{document}

\title{Problem Set 2}
\date{Due: February 6th, 5pm}
\maketitle


For each problem, you must give a complete and rigorous proof. Your arguments should be clear, logically ordered, and written in full sentences.

In particular, state all assumptions explicitly and define all notation. Justify every nontrivial step. 

\begin{questions}

\question Prove that the product of two odd integers is odd. 

% Contrapositive
\question Prove that if $n^2$ is divisible by $3$, then $n$ is divisible by $3$.
%Test: If $n^2$ is even, then $n$ is even. 


% Existential quantifiers
\question Prove or disprove the following statements.
\begin{parts}
    \part $\forall x \in \mbb{R}\,  \exists y \in \mbb{R} : x^2 > y$
    \part $\exists y \in \mbb{R}\,  \forall x \in \mbb{R} : x^2 > y$. 
    \part (\emph{Hint: You may use results that we have proven in class. }) 
        \[ \forall a \in \mbb{Z}, \forall n \in \mbb{Z}_{> 0}: \left(\gcd(a,n) = 1 \implies (\exists x \in \mbb{Z} \colon n \mid 1 - ax)\right). \]

\end{parts}

\question Prove that for all integer $n \geq 1$, the following formula holds
    \[ 1\cdot 2 + 2 \cdot 3 + \cdots + n\cdot(n+1) = \frac{n(n+1)(n+2)}{3}. \]


% Strong induction
\question Prove that every integer $n > 1$ is either prime or can be written as a product of prime numbers. \emph{Hint: Use strong induction. }

% Direct proof. Induction. Contradiction
\question Let $n$ be an odd integer. We say that $n$ is Type 1 (respectively Type 3) if $n$ can be written as $4k + 1$ (respectively $4k + 3$) for some integer $k$. 
\begin{parts}
    \part Let $n, m$ be odd integers. Consider the statement  ``$nm$ is a Type 3 integer only if \emph{exactly one} of $n, m$ is a Type 3 integer". Write this as an ``If... then..." statement, and give a direct proof, using only the definition. 
    \part Prove by induction the statement that if $x_1, \ldots, x_n$ are odd integers such that the product $\displaystyle \prod_{i =1}^n x_i = x_1\cdot x_2 \cdots x_n$ is a Type 3 integer, then \emph{at least} one of $x_1, x_2, \ldots, x_n$ is a Type 3 integer. 
    \part Prove that there are infinitely many prime numbers that are Type 3. \emph{Hint: Modify Euclid's proof that there are infinitely many primes by considering numbers $4p_1\cdots p_n + 3$. }
\end{parts}

% Cases. 
\question Let $i$ be an integer, and let $P_i$ be the statement ``There are infinitely prime numbers of the form $3k + i$ for some integer $k$".
\begin{parts}
    \part Prove that $P_i \implies P_{i + 3}$. 
    \part Write down the statement $P_1 \lor P_2$ (where $\lor$ is the logical or) as a sentence, and prove it. In your proof, you may use that there are infinitely many prime numbers. 
\end{parts}



\end{document}