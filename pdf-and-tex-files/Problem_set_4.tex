\documentclass[12pt]{exam}


% Symbols
\usepackage{amsmath, amsthm, amssymb}
\usepackage{mathtools}
\usepackage{wasysym}
\usepackage{mathrsfs}  
\usepackage{stmaryrd}
\usepackage{comment}
\usepackage{marginnote}

% Style
\usepackage[margin=1in]{geometry}
\usepackage{parskip}
\usepackage[shortlabels]{enumitem}

% Colored text
%\usepackage{color}
\usepackage{xcolor}

% Links
\usepackage[bookmarks,colorlinks,breaklinks]{hyperref}
% PDF hyperlinks, with coloured links
\hypersetup{linkcolor=red,citecolor=blue,
  filecolor=dullmagenta,urlcolor=blue}

% Bibliography and references
\usepackage[numbers]{natbib}
\usepackage[capitalize]{cleveref}

% Figures
\usepackage{graphicx}
\usepackage{tikz}
\usetikzlibrary{matrix, arrows}
\usetikzlibrary{cd}

\NeedsTeXFormat{LaTeX2e}
\ProvidesPackage{quiver}[2021/01/11 quiver]

% `tikz-cd` is necessary to draw commutative diagrams.
\RequirePackage{tikz-cd}
% `amssymb` is necessary for `\lrcorner` and `\ulcorner`.
\RequirePackage{amssymb}
% `calc` is necessary to draw curved arrows.
\usetikzlibrary{calc}
% `pathmorphing` is necessary to draw squiggly arrows.
\usetikzlibrary{decorations.pathmorphing}

% A TikZ style for curved arrows of a fixed height, due to AndréC.
\tikzset{curve/.style={settings={#1},to path={(\tikztostart)
    .. controls ($(\tikztostart)!\pv{pos}!(\tikztotarget)!\pv{height}!270:(\tikztotarget)$)
    and ($(\tikztostart)!1-\pv{pos}!(\tikztotarget)!\pv{height}!270:(\tikztotarget)$)
    .. (\tikztotarget)\tikztonodes}},
    settings/.code={\tikzset{quiver/.cd,#1}
        \def\pv##1{\pgfkeysvalueof{/tikz/quiver/##1}}},
    quiver/.cd,pos/.initial=0.35,height/.initial=0}
\usepackage[all]{xy}
   \SelectTips{cm}{10}

\usepackage{hyperref}
\usepackage{enumitem}
\usepackage{graphicx}

\makeatletter
\def\thm@space@setup{%
  \thm@preskip=0.5em\thm@postskip=\thm@preskip%
}
\makeatother
% Unnamed theorem without counter
\newtheorem*{theorem}{Theorem}
\newtheoremstyle{named}{}{}{\\itshape}{}{\bfseries}{.}{.5em}{\thmnote{#3's }#1}
% Named theorem without counter
\theoremstyle{named}
\newtheorem*{namedtheorem}{Theorem}
% Other theorem environments
\theoremstyle{plain}
\newtheorem{thm}{Theorem}[section]
\newtheorem{conj}[thm]{Conjecture}
\newtheorem{prop}[thm]{Proposition}
\newtheorem{lem}[thm]{Lemma}
\newtheorem{sublem}[thm]{Sub-Lemma}
\newtheorem{cor}[thm]{Corollary}
\theoremstyle{definition}
\newtheorem{defn}[thm]{Definition}
\newtheorem{ex}[thm]{Exercise}
\newtheorem{eg}[thm]{Example}
\newtheorem{assumption}[thm]{Assumption}
\newtheorem{convention}[thm]{Convention}
\newtheorem{fact}{Fact}
\crefformat{footnote}{#2\footnotemark[#1]#3}

%Hom
\newcommand{\Hom}{\mathrm{Hom}}

% Fonts and notation
\newcommand{\CC}{\mathbb{C}}
\newcommand{\QQ}{\mathbb{Q}}
\newcommand{\RR}{\mathbb{R}}
\newcommand{\ZZ}{\mathbb{Z}}

\newcommand{\mf}[1]{\mathfrak{#1}}
\newcommand{\mc}[1]{\mathcal{#1}}
\newcommand{\ms}[1]{\mathscr{#1}}
\newcommand{\mr}[1]{\mathrm{#1}}
\newcommand{\mbb}[1]{\mathbb{#1}}
\newcommand{\ol}[1]{\overline{#1}}
\newcommand{\ul}[1]{\underline{#1}}
\newcommand{\wt}[1]{\widetilde{#1}}
\newcommand{\wh}[1]{\widehat{#1}}

% Geometry
\newcommand{\Spa}{\mathrm{Spa}}
\newcommand{\Spec}{\mathrm{Spec}}
\newcommand{\Spf}{\mathrm{Spf}}
\newcommand{\Spd}{\mathrm{Spd}}

\newcommand{\an}{\mathrm{an}}

% sites
\newcommand{\proet}{\mr{pro\acute{e}t}}
\newcommand{\et}{\mr{\acute{e}t}}
\newcommand{\fet}{\mr{f\acute{e}t}}


%Algebraic groups
\newcommand{\GL}{\mathrm{GL}}
\newcommand{\SL}{\mathrm{SL}}
\newcommand{\SO}{\mathrm{SO}}
\newcommand{\OO}{\mathrm{O}}
\newcommand{\Sp}{\mathrm{Sp}}
\newcommand{\GSp}{\mathrm{GSp}}
\newcommand{\Spin}{\mathrm{Spin}}
\newcommand{\GSpin}{\mathrm{GSpin}}
\newcommand{\SU}{\mathrm{SU}}
\newcommand{\Stab}{\mathrm{Stab}}
\newcommand{\Rep}{\mathrm{Rep}}

% Miscellaneous
\newcommand{\Frob}{\mathrm{Frob}}
\DeclareMathOperator{\Aut}{Aut}
\newcommand{\colim}{\mathrm{colim}}
\renewcommand{\l}{\left}
\renewcommand{\r}{\right}

% p-adic geom
\newcommand{\crys}{\mathrm{crys}}
\newcommand{\Fil}{F}
\newcommand{\HT}{\mathrm{HT}}
\newcommand{\Gr}{\mathrm{Gr}}
\newcommand{\Fl}{\mathrm{Fl}}
\newcommand{\cris}{\mathrm{cris}}
\newcommand{\Isoc}{\mr{Isoc}} 
\newcommand{\dR}{\mathrm{dR}}
\newcommand{\Perf}{\mathrm{Perf}}
\newcommand{\Perfd}{\mathrm{Perfd}}
\newcommand{\FF}{\mathrm{FF}}


% Shimura varieties
\newcommand{\Sh}{\mathrm{Sh}}

% Coefficient objects for Weil cohomology theories
\newcommand{\Loc}{\mathrm{Loc}}
\newcommand{\MIC}{\mathrm{MIC}}

%Spacing
\newcommand{\+}{\, }


\begin{document}

\title{Problem Set 4}
\date{Due: February 28th, 5pm}
\maketitle





\section{Formal proofs}


For each problem in this section, you must give a complete and rigorous proof. Your arguments should be clear, logically ordered, and written in full sentences.

In particular, state all assumptions explicitly and define all notation. Justify every nontrivial step. 

\begin{questions}
\question[1] Binomial Coefficients

\begin{parts}

\part For integers $n \ge 0$ and $0 \le k \le n$, the binomial coefficient is defined by
\[
\binom{n}{k} = \frac{n!}{k!(n-k)!}.
\]
Compute $\binom{4}{0}$, $\binom{4}{1}$, $\binom{4}{2}$, $\binom{4}{3}$, and $\binom{4}{4}$.

\part Prove that for all $0 \le k \le n$,  $\binom{n}{k} = \binom{n}{n-k}.$

\part Let $n \geq 1$. Prove that for $1 \le k \le n-1$, the binomial coefficients satisfy Pascal's identity: $
\binom{n}{k} = \binom{n-1}{k} + \binom{n-1}{k-1}.$ Look up Pascal's triangle, and write down the first 5 lines. 

\end{parts}


\question[2] The Binomial Theorem. 

\begin{parts}

\part Expand $(x+y)^2$, $(x+y)^3$, and $(x+y)^4$ by direct multiplication. Rewrite your expansions so that the coefficients appear as binomial coefficients.
\part Prove by induction on $n$ the Binomial Theorem
\[
(x+y)^n = \sum_{k=0}^n \binom{n}{k} x^{\,n-k} y^k.
\]

\part Let $p$ be prime and $x,y \in \ZZ$. Prove that $(x+y)^p \equiv x^p + y^p \pmod{p}$. 

\end{parts}

\section{Demonstrations}
\fullwidth{For problems in this section, you still need to give complete mathematical reasoning to support your answers. I should be able to follow and understand your work, but it doesn't have to be organized into a formal proof. }


\question[1] Solve the simultaneous congruence equations
\[
\begin{cases}
3x \equiv 4 &\pmod{17} \\
8x + 11 \equiv 5 &\pmod{31}
\end{cases}
\]

\question[1] Solve the simultaneous congruence equations
\[
\begin{cases}
x \equiv 2 \pmod{5}, \\
x \equiv 3 \pmod{7}, \\
x \equiv 4 \pmod{9}.
\end{cases}
\]

\question[1] How many units are there in $\mathbb{Z}_{2600}$? 

\question[2]
If $a$ is a positive integer, we can write $a$ uniquely as 
	\[ a = \sum_{i = 0}^n d_i 2^i = d_n\cdot 2^n + d_{n-1}\cdot 2^{n-1} + \cdots + d_1\cdot 2 + d_0\cdot 2^0 \qquad d_i \in \{0,1\}. \]
The digits $d_n \cdots d_0$ are the \textbf{binary digits} of $a$, and we write $(a)_2$ for the binary digits. For example $(11)_2 = 1011$ since
	\[ 11 =   8 + 2 + 1 = 1\cdot 2^3 + 0\cdot2^2 + 1\cdot 2^1 + 1\cdot 2^0. \]

\begin{parts}

\part Find the binary expansion of $1345$.  

\part Find $7^{2^i} \pmod{81}$ for $i = 0,1\ldots, 10$. 

\part Use your results from parts (a) and (b) to compute $7^{1345} \pmod{81}$ efficiently. 

\end{parts}

\newpage

\section{Explorations}
\fullwidth{In this section, I'm looking for answers that are supported by evidence, but you don't have to prove your answers. }

\question[2] Let $n$ be a positive integer, and let $\mathbb{Z}_n^\times = \{ x \in \mathbb{Z}_n \colon \text{$x$ is a unit modulo $n$}\}$. Let 
	\[ O_n = \{k \colon k = \text{ord}_n(a) \text{ for some } a \in \mathbb{Z}_n^\times\}. \]
\begin{parts}
	\part Write down the sets $O_n$ for $n = 1, \ldots, 100$. (\emph{You are of course welcome to do this problem by hand, but I wouldn't recommend it \smiley{}. Instead, I recommend that you use SageMath, in which case you need to provide screen shots of both your code and the results you have found.})
	\part Give an upper bound for the biggest possible value in $O_n$, and provide some explanation of why this is an upper bound.
	\part For which values of $n$ from part (a) does $O_n$ achieve the upper bound you have given in part (b)? 
	\part Write down some observations about the sets $O_n$. Do you have any conjectures about the structure of this sets? E.g., are there cases where you can determine exactly which numbers appear in $O_n$ based on just $n$ alone? 
\end{parts}

\end{questions}
\end{document}