\documentclass[12pt]{exam}


% Symbols
\usepackage{amsmath, amsthm, amssymb}
\usepackage{mathtools}
\usepackage{wasysym}
\usepackage{mathrsfs}  
\usepackage{stmaryrd}
\usepackage{comment}
\usepackage{marginnote}

% Style
\usepackage[margin=1in]{geometry}
\usepackage{parskip}
\usepackage[shortlabels]{enumitem}

% Colored text
%\usepackage{color}
\usepackage{xcolor}

% Links
\usepackage[bookmarks,colorlinks,breaklinks]{hyperref}
% PDF hyperlinks, with coloured links
\hypersetup{linkcolor=red,citecolor=blue,
  filecolor=dullmagenta,urlcolor=blue}

% Bibliography and references
\usepackage[numbers]{natbib}
\usepackage[capitalize]{cleveref}

% Figures
\usepackage{graphicx}
\usepackage{tikz}
\usetikzlibrary{matrix, arrows}
\usetikzlibrary{cd}

\NeedsTeXFormat{LaTeX2e}
\ProvidesPackage{quiver}[2021/01/11 quiver]

% `tikz-cd` is necessary to draw commutative diagrams.
\RequirePackage{tikz-cd}
% `amssymb` is necessary for `\lrcorner` and `\ulcorner`.
\RequirePackage{amssymb}
% `calc` is necessary to draw curved arrows.
\usetikzlibrary{calc}
% `pathmorphing` is necessary to draw squiggly arrows.
\usetikzlibrary{decorations.pathmorphing}

% A TikZ style for curved arrows of a fixed height, due to AndréC.
\tikzset{curve/.style={settings={#1},to path={(\tikztostart)
    .. controls ($(\tikztostart)!\pv{pos}!(\tikztotarget)!\pv{height}!270:(\tikztotarget)$)
    and ($(\tikztostart)!1-\pv{pos}!(\tikztotarget)!\pv{height}!270:(\tikztotarget)$)
    .. (\tikztotarget)\tikztonodes}},
    settings/.code={\tikzset{quiver/.cd,#1}
        \def\pv##1{\pgfkeysvalueof{/tikz/quiver/##1}}},
    quiver/.cd,pos/.initial=0.35,height/.initial=0}
\usepackage[all]{xy}
   \SelectTips{cm}{10}

\usepackage{hyperref}
\usepackage{enumitem}
\usepackage{graphicx}

\makeatletter
\def\thm@space@setup{%
  \thm@preskip=0.5em\thm@postskip=\thm@preskip%
}
\makeatother
% Unnamed theorem without counter
\newtheorem*{theorem}{Theorem}
\newtheoremstyle{named}{}{}{\\itshape}{}{\bfseries}{.}{.5em}{\thmnote{#3's }#1}
% Named theorem without counter
\theoremstyle{named}
\newtheorem*{namedtheorem}{Theorem}
% Other theorem environments
\theoremstyle{plain}
\newtheorem{thm}{Theorem}[section]
\newtheorem{conj}[thm]{Conjecture}
\newtheorem{prop}[thm]{Proposition}
\newtheorem{lem}[thm]{Lemma}
\newtheorem{sublem}[thm]{Sub-Lemma}
\newtheorem{cor}[thm]{Corollary}
\theoremstyle{definition}
\newtheorem{defn}[thm]{Definition}
\newtheorem{ex}[thm]{Exercise}
\newtheorem{eg}[thm]{Example}
\newtheorem{assumption}[thm]{Assumption}
\newtheorem{convention}[thm]{Convention}
\newtheorem{fact}{Fact}
\crefformat{footnote}{#2\footnotemark[#1]#3}

%Hom
\newcommand{\Hom}{\mathrm{Hom}}

% Fonts and notation
\newcommand{\CC}{\mathbb{C}}
\newcommand{\QQ}{\mathbb{Q}}
\newcommand{\RR}{\mathbb{R}}
\newcommand{\ZZ}{\mathbb{Z}}

\newcommand{\mf}[1]{\mathfrak{#1}}
\newcommand{\mc}[1]{\mathcal{#1}}
\newcommand{\ms}[1]{\mathscr{#1}}
\newcommand{\mr}[1]{\mathrm{#1}}
\newcommand{\mbb}[1]{\mathbb{#1}}
\newcommand{\ol}[1]{\overline{#1}}
\newcommand{\ul}[1]{\underline{#1}}
\newcommand{\wt}[1]{\widetilde{#1}}
\newcommand{\wh}[1]{\widehat{#1}}

% Geometry
\newcommand{\Spa}{\mathrm{Spa}}
\newcommand{\Spec}{\mathrm{Spec}}
\newcommand{\Spf}{\mathrm{Spf}}
\newcommand{\Spd}{\mathrm{Spd}}

\newcommand{\an}{\mathrm{an}}

% sites
\newcommand{\proet}{\mr{pro\acute{e}t}}
\newcommand{\et}{\mr{\acute{e}t}}
\newcommand{\fet}{\mr{f\acute{e}t}}


%Algebraic groups
\newcommand{\GL}{\mathrm{GL}}
\newcommand{\SL}{\mathrm{SL}}
\newcommand{\SO}{\mathrm{SO}}
\newcommand{\OO}{\mathrm{O}}
\newcommand{\Sp}{\mathrm{Sp}}
\newcommand{\GSp}{\mathrm{GSp}}
\newcommand{\Spin}{\mathrm{Spin}}
\newcommand{\GSpin}{\mathrm{GSpin}}
\newcommand{\SU}{\mathrm{SU}}
\newcommand{\Stab}{\mathrm{Stab}}
\newcommand{\Rep}{\mathrm{Rep}}

% Miscellaneous
\newcommand{\Frob}{\mathrm{Frob}}
\DeclareMathOperator{\Aut}{Aut}
\newcommand{\colim}{\mathrm{colim}}
\renewcommand{\l}{\left}
\renewcommand{\r}{\right}

% p-adic geom
\newcommand{\crys}{\mathrm{crys}}
\newcommand{\Fil}{F}
\newcommand{\HT}{\mathrm{HT}}
\newcommand{\Gr}{\mathrm{Gr}}
\newcommand{\Fl}{\mathrm{Fl}}
\newcommand{\cris}{\mathrm{cris}}
\newcommand{\Isoc}{\mr{Isoc}} 
\newcommand{\dR}{\mathrm{dR}}
\newcommand{\Perf}{\mathrm{Perf}}
\newcommand{\Perfd}{\mathrm{Perfd}}
\newcommand{\FF}{\mathrm{FF}}


% Shimura varieties
\newcommand{\Sh}{\mathrm{Sh}}

% Coefficient objects for Weil cohomology theories
\newcommand{\Loc}{\mathrm{Loc}}
\newcommand{\MIC}{\mathrm{MIC}}

%Spacing
\newcommand{\+}{\, }


\begin{document}

\title{Problem Set 3}
\date{Due: February 13th, 5pm}
\maketitle


\begin{questions}

\section{Formal proofs}
For each problem in this section, you must give a complete and rigorous proof. Your arguments should be clear, logically ordered, and written in full sentences.

In particular, state all assumptions explicitly and define all notation. Justify every nontrivial step. 

\question[1] Let $s, t$ be positive odd integers such that $s > t$ and $s,t$ are relatively prime (i.e.\ $\gcd(s,t) = 1$). Show that 
    \[ st, \quad \frac{s^2 - t^2}{2}, \quad \text{ and } \quad \frac{s^2 + t^2}{2} \]
are {pairwise relatively prime} (i.e.\ the $\gcd$ of any two of them is $1$). 


\question[1] Prove that $\sqrt{3}$ is irrational. 

\question[1] Prove that $\log_{10}(5)$ is irrational 

\question[2] Suppose that $x,y, z$  are positive integers such that $\gcd(x,y,z) = 1$ and satisfy
    \[ x^2 + 2y^2 = z^2. \]
\begin{parts}
    \part Prove that $x,z$ have the same parity (i.e. are both either even or odd). \emph{Hint: Rearrange into an equation $2y^2 = z^2 - x^2$.}
    \part Prove that $y$ is even. 
\end{parts}

\newpage 

\section{Demonstrations}
For problems in this section, you still need to give complete mathematical reasoning to support your answers. I should be able to follow and understand your work, but it doesn't have to be organized into a formal proof. 

\question[3] Let $\ZZ[\sqrt{-5}] = \{a + b\sqrt{-5} \colon a,b \in \ZZ\} \subseteq \CC$, and consider the function
    \[ N \colon \ZZ[\sqrt{-5}] \to \ZZ \qquad N(a + b \sqrt{-5}) = a^2 + 5b^2 \]
\begin{parts}
    \part Show that if $x, y \in \ZZ[\sqrt{-5}]$ then both $x + y$ and $x\cdot y$ are in $\ZZ[\sqrt{-5}]$. 
    \part Show that $N$ satisfies the following properties:
            \begin{enumerate}[(i)]
                \item $N(x) \geq 0$ for all $x \in \ZZ[\sqrt{-5}]$ and $N(x) = 0$ if and only if $x = 0$.
                \item $N(xy) = N(x)N(y)$ for all $x,y \in \ZZ[\sqrt{-5}]$. 
            \end{enumerate}
    \part An element $x \in \ZZ[\sqrt{-5}]$ is a called a \emph{unit} if there exists $y$ such that $x\cdot y = 1$. Show that $x$ is a unit if and only if $N(x) = 1$, and use this to find all units in $\ZZ[\sqrt{-5}]$.  
    \part An element $p \in \ZZ[\sqrt{-5}]$ is called \emph{prime} if whenever $p = x\cdot y$ for some $x, y \in \ZZ[\sqrt{-5}]$, either $x$ or $y$ is a unit. 
    \part Determine which of the elements $2, 3, 5, 7, 1 + \sqrt{-5}, 2 - 3\sqrt{-5}$ are prime. %Show that the elements $2, 3, 1 + \sqrt{-5}$ and $1-\sqrt{-5}$ are 
    \part Show that $6$ does not have a unique factorization into primes. (Part of this exercise is for you to formulate what unique prime factorization means in $\ZZ[\sqrt{-5}]$!) 
\end{parts}

\section{Explorations}
In this section, I'm looking for answers that are supported by evidence, but you don't have to prove your answers. 

\question[1] Consider the equation $6x + 15y + 27z = d$. 
\begin{parts} 
    \part What are all of the integer solutions when $d = 12$? 
    \part Make a conjecture to answer the following question: For which $d \in \ZZ$ does $6x + 15y + 27z = d$ have a solution (in the integers)?
\end{parts}

\question[1] Let $p > 3$ be a prime integer. When does $p$ remain prime as an element of $\ZZ[\sqrt{-5}]$? Compile a list of primes up to $100$, and determine if they remain prime in $\ZZ[\sqrt{-5}]$. Do you have a conjecture for which $p$ remain prime? \emph{Hint:  Talk to me if you need help getting started, and I strongly recommend using SageMath to do this.} 

\question (Bonus) \emph{This question is optional, it will not be graded and is here for fun \smiley{}.}  Generalized the above exercise to $\ZZ[\sqrt{d}]$ for $d \in \ZZ$ squarefree. 
\end{questions}
\end{document}