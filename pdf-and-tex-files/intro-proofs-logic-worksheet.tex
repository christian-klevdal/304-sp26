\documentclass[11pt]{article}
\input{../AdamMathPreamble2.sty}
%\usepackage{AdamMathPreamble}
\usepackage{pdfpages}
\usepackage{subfiles}
%\usepackage{pgfplots}
\usepackage{mdframed,xcolor,paracol}
\newcommand{\cb}{\columnbreak}
\newcommand{\FRB}[1]{\bigskip\begin{mdframed}[backgroundcolor=blue!20]#1
    \end{mdframed}}
\newcommand{\fireside}[1]{\columnratio{0.33,0.66}
\begin{paracol}{2}
%\CG{2in}{fireplace.jpeg}
\switchcolumn
(Fireside Chat)#1
\end{paracol}
}

\pagestyle{empty}
\title{Introduction to Logic and Proofs}
\usepackage{mdframed,xcolor}
\date{}
%\includeversion{sol}\excludeversion{nosol}
\includeversion{nosol}\excludeversion{sol}

\begin{document}

\section*{Introduction to Logic and Proofs}

\FR{In this worksheet we will cover the fundamentals of mathematical logic and proof techniques:
\begin{itemize}
\item Propositional logic: statements, negations, implications, converse, and contrapositive
\item Quantifiers: existential and universal, and their negations
\item Direct proof
\item Proof by contrapositive
\item Proof by contradiction

\item Mathematical induction
\end{itemize}
}


\noindent{\bf Guide:} In this worksheet, boxed sections with gray/blue backgrounds contain definitions and examples that I will present. Sections labeled ``{\bf Exercise}'' or ``{\bf Your Turn}'' are problems for you to work on.

\section{Propositional Logic}

\FRB{In mathematical logic we use several important symbols:

{\bf $\lor$} which means \uline{\hskip 1in} (OR);
 
{\bf $\land$} which means \uline{\hskip 1in} (AND);

{\bf $\neg$} which means \uline{\hskip 1in} (NOT/negation);

{\bf $\implies$} which means \uline{\hskip 1in} (implies/if...then);

{\bf Note:} In this class, OR is always {\bf inclusive}: ``$p$ OR $q$'' is true if either or both statements are true.
}

\subsection{Statements and Evaluating Propositions} 
{\bf Example:} Below is a list of propositions. Discuss the truth value of each:
\begin{enumerate}
\item $7<\pi$ or $10<3^2$.
\item It is not the case that $7<\pi$.
\item $10<3^2$ or it is not the case that $7<\pi$.
\item $2^9=512$ and it is not the case that $10<3^2$.
\item It is not the case that $7<\pi$ or $10<3^2$.
\end{enumerate}

{\bf Exercise:} {Translating Between Math and English}

Let $p$ and $q$ denote the statements ``We won our first game'' and ``We won our second game.''  Rewrite each using the symbols $p, q,\land, \lor, \neg$:
\begin{enumerate}[a.)]
\item We won both of our first two games. \medskip 
\item We lost both of our first two games.\medskip  
\item We won at least one of our first two games.\medskip  
\item We lost at least one of our first two games.\medskip  
\item We didn't win both of our first two games. \medskip
\end{enumerate}

{Two of the above are {\bf logically equivalent}. Which two?}

\newpage
\subsection{Implications: Conditional Statements, Converse and Contrapositive.}

Sometimes we want to make {\bf conditional} statements. For example:  ``If $x$ is a whole number then $4x$ is an even number.''  

These are examples of ``implications'' or ``conditional statements.'' We denote such statements as 
$p \implies q$, where $p$ is called the {\bf hypothesis} and $q$ is called the {\bf conclusion.} 

\FRB{
The {\bf converse} of $p \implies q$ is \uline{\hskip 1.5in}

The {\bf contrapositive} of $p \implies q$ is \uline{\hskip 1.5in}

The contrapositive is always logically equivalent to the original statement, while the converse is not!
}

\smallskip

{\bf Story Time:} Understanding implications

\begin{center}$p \implies q$:  ``If it's Tuesday, then I go rock climbing.''\end{center}

Under which of the following situations would you say that this sentence is FALSE? (In other words, when would you call me out as a liar?) 

It's Tuesday and I go rock climbing. \hfill [Telling the truth / Liar]\medskip 

It's Tuesday and I do not go rock climbing.\hfill [Telling the truth / Liar]\medskip   

It's not Tuesday and I go rock climbing.\hfill [Telling the truth / Liar]\medskip   

It's not Tuesday and I do not go rock climbing. \hfill [Telling the truth / Liar]\medskip   



Which of the following are correct reformulations of ``If it's Tuesday then I go rock climbing''?
\begin{itemize}
	\item I climb every Tuesday
	\item I only climb on Tuesdays
	\item I don't climb on Wednesday
	\item Every Tuesday I climb
	\item If I climb, then it's Tuesday
\end{itemize}

\FR{
\vspace{0.5cm}

$p \implies q$ can be rewritten in terms of $\neg, \lor, \land$ as \uline{\hskip 1.5in}. \\

To prove $p \implies q$:I need to assume that $p$ is \textbf{True / False} (circle one) and prove that $q$ is \textbf{True / False}. 
}


{\bf Exercise:} For each statement, identify the hypothesis and conclusion, then write the converse and contrapositive:

\begin{enumerate}[a.)]
\item All birds lay eggs

(First rewrite as ``IF \uline{\hskip 2in} THEN \uline{\hskip 2in}'')

Hypothesis: \hfill Conclusion: \hfill \,

Converse: \hfill Contrapositive: \hfill \, 


\item Only California residents live in San Diego

(First rewrite as ``IF \uline{\hskip 2in} THEN \uline{\hskip 2in}'')

Hypothesis: \hfill Conclusion: \hfill \, 

Converse: \hfill Contrapositive: \hfill \, 
\end{enumerate}

\newpage
\subsection{Negation and DeMorgan's Laws}

Sometimes a statement might be hard to wrap your head around. Consider:

``It is not the case that either Alice is friends with Bob or that if Bob is friends with Charlie then Dave has no friends." 

To simplify such statements, we need to understand negation. Let's build intuition:

{\bf Exercise:} Write the following statements as a statement beginning with "I do not like..." 

``It is not the case that (I like chocolate or tea).''  

\vspace{1cm}

``It is not the case that (I like chocolate and tea).''  

\vspace{1cm}


\FR{
These are examples of {\bf DeMorgan's Laws:}  
$$\neg (p \land q) = \neg p \lor \neg q$$
$$\neg (p \lor q) = \neg p \land \neg q$$
These laws help us simplify compound statements by moving negations inside parentheses.}

\bigskip

{\bf Exercise :} Negate the following statements using DeMorgan's Laws:
\begin{enumerate}
\item $p \land q$
\item $(p \lor q) \land r$
\item $p \implies q$ (Hint: recall that $p \implies q$ is equivalent to $\neg p \lor q$)
\end{enumerate}

\section{Quantifiers}

In mathematics, we often make statements about collections of objects. We use quantifiers to express these ideas.

\FRB{
The {\bf universal quantifier} $\forall$ means ``for all'' or ``for every''

The {\bf existential quantifier} $\exists$ means ``there exists'' or ``there is at least one''
}

\subsection{Examples of Quantified Statements}

\begin{enumerate}
\item $\forall x \in \mathbb{R}, x^2 \geq 0$ means: ``For all real numbers $x$, $x^2$ is non-negative.''

\item $\exists x \in \mathbb{Z}, x^2 = 4$ means: ``There exists an integer $x$ such that $x^2 = 4$.''

\item $\forall n \in \mathbb{N}, \exists m \in \mathbb{N}, m > n$ means: ``For every natural number $n$, there exists a natural number $m$ such that $m > n$.''
\end{enumerate}

\subsection{Negating Quantifiers}

Understanding how to negate quantified statements is crucial:

\FRB{
Rules for negating quantifiers:
\begin{itemize}
\item \qquad $\neg(\forall x, P(x)) \equiv \exists x, \neg P(x)$. \qquad In words: the negation of ``for all'' is ``there exists...not''
\item \qquad $\neg(\exists x, P(x)) \equiv \forall x, \neg P(x)$.  \qquad In words:  the negation of ``there exists'' is ``for all...not''
\end{itemize}
}

{\bf Exercise:} Negate the following statements:

\begin{enumerate}
\item For all integers $n$, $n^2$ is even.
\vspace{0.5cm}

\item There exists a real number $x$ such that $x^2 < 0$.
\vspace{0.5cm}
\item For all students in this class, there exists a problem on the homework that they can solve.
\vspace{0.5cm}
\end{enumerate}


\section{Proof Techniques}

Now that we understand logical statements, let's learn how to prove them!
\subsection{Direct Proof}

\FRB{
In a {\bf direct proof}, we start with the hypothesis and proceed step by step to the conclusion.

{\bf Strategy:}
\begin{itemize}
\item Assume the hypothesis is true
\item Use definitions, known facts, and logical reasoning
\item Arrive at the conclusion
\end{itemize}
}

\subsection{Example 1}

{\bf Proposition:} If $x$ and $y$ are integers and $x + y$ is even then $x - y$ is even.

\begin{proof}
Suppose $x,y$ are integers such that $x + y$ is even. 

By definition of an even integer, this means that $x+ y = 2k$ for some $k \in \mathbb{Z}$. Consequently, 
	\[ x- y = (x + y) - 2y = 2k - 2y = 2(k - y). \]
Since $k-y$ is also an integer, we conclude that $x-y$ is even. 
\end{proof}

{\bf Notes:} The above proof is well written: our assumptions are clearly stated, every variable has been identified and introduced correctly (i.e.\ what set does $k$ live in), every step is justified, and we clearly arrive at the desired conclusion. And very importantly: complete, grammatically correct sentences have been used! 

\subsection{Example 2}

{\bf Proposition:} If $m$ is an even positive integer then $(m + 1)^2$ is odd.

\begin{proof}
\, 
\vspace{2cm}

\, 
\end{proof}

\newpage

\subsection{Exercise: Direct Proofs}

Write direct proofs for the following:

\begin{enumerate}
\item If $k$ is a positive integer such that $5k - 1$ is even, then $9k + 13$ is even. 

\vspace{3cm}

\item Suppose $n$ and $m$ are integers and $n \neq 0$. Prove that if $n | m$ then $n | (m^2 + m)$.  

\vspace{3cm}

\item Prove that if $r$ and $s$ are rational numbers then so is $r + s$. 

\vspace{3cm}

\end{enumerate}

\subsection{Proof by Contrapositive}
Sometimes a direct proof is difficult, but the contrapositive is easier to prove. 

\FRB{

Recall: To prove $P \implies Q$, we can instead prove the contrapositive: $\neg Q \implies \neg P$

These are {\bf logically equivalent}, so proving one proves the other!

{\bf When to use contrapositive:}
\begin{itemize}
\item When the hypothesis is easier to work with than the conclusion
\item When you need to prove something is NOT true
\item When the contrapositive gives you more to work with
\end{itemize}
}


{\bf Example:} Let $\beta$ be an irrational number. Prove that $\beta - 4$ is also irrational.

{\bf Strategy:} The direct approach would be ``Assume $\beta$ is irrational, prove $\beta - 4$ is irrational.'' But it's hard to work with irrationality directly! Instead, prove the contrapositive: ``If $\beta - 4$ is rational, then $\beta$ is rational.''
\begin{proof}
\, 

\vspace{3cm}

\, 
\end{proof}


\bigskip

{\bf Your Turn:} Use proof by contrapositive:

\begin{enumerate}
\item Prove that if $x^3 + 5x = 40$ then $x < 3$.

\vspace{3cm}

\item Prove that if $a$ is not rational then $a + 7$ is not rational. 

\vspace{3cm}

\end{enumerate}

\subsection{Proof by Contradiction}

\FRB{
{\bf Proof by Contradiction} (Reductio Ad Absurdum):

\begin{itemize}
\item Assume the statement you want to prove is FALSE
\item Use logical reasoning to derive a contradiction (e.g., $1 = 0$ or ``$x$ is both even and odd'')
\item Conclude that your assumption must be wrong, so the statement is TRUE
\end{itemize}
}

\subsection{Example 1}

{\bf Proposition:} There does not exist a largest natural number. 

\begin{proof}
We argue by contradiction. Suppose that $N$ is the largest natural number. This means $N \geq m$ for every natural number $m$.

But consider $N + 1$. This is also a natural number (since natural numbers are closed under addition). 

If $N$ was greater than or equal to {\bf every} natural number, then $N \geq N + 1$.

Subtracting $N$ from both sides gives $0 \geq 1$, which is false.

We have reached a contradiction, so there can be no largest natural number.
\end{proof}

\subsection{Example 2}

{\bf Proposition:} There is no smallest positive rational number.

\begin{proof}
We argue by contradiction. Suppose there were a smallest positive rational number, call it $x$.

Then $x > 0$ and for every positive rational number $r$, we have $x \leq r$.

Consider $\frac{x}{2}$. Since $x$ is rational (say $x = \frac{m}{n}$), we have $\frac{x}{2} = \frac{m}{2n}$, which is also rational.

Also, $\frac{x}{2} > 0$ since $x > 0$.

But $\frac{x}{2} < x$, which contradicts our assumption that $x$ was the smallest positive rational number.

Therefore, there is no smallest positive rational number. 
\end{proof}

\subsection{Exercise: Contradiction Proofs}

Use proof by contradiction:

\begin{enumerate}
\item Your friend claims there is a perfect square greater than 10 that is exactly one more than a prime number. Show this is incorrect.

\vspace{5cm}

\item Prove Euclid's theorem: There are infinitely many prime numbers. 
\vfill

\end{enumerate}


\section{If and only if statements}
\FRB{
An ``if and only if'' statement has the form $P \iff Q$, which means:
\begin{itemize}
\item $P \implies Q$ (if $P$ then $Q$), AND
\item $Q \implies P$ (if $Q$ then $P$)
\end{itemize}

To prove an if and only if statement, you must prove BOTH directions!
}

\newpage

\subsection{Example}

{\bf Proposition:} Let $a$ be an integer. Then $a$ is even if and only if $a^2$ is even.

\begin{proof}
There are two parts.

{\bf First direction:} We show that if $a$ is even, then $a^2$ is even. \vspace{2cm}

{\bf Second direction:} We show that if $a^2$ is even, then $a$ is even. \vspace{2.5cm}

We have proven both directions, so the if and only if statement is true.
\end{proof}

{\bf{Question}:} 
\begin{itemize}
\item The statement ``$a$ is even if $a^2$ is even" is proven in the first or second direction above?  \hfill \textbf{First}\, / \, \textbf{Second}
\item What about the statement ``$a$ is even only if $a^2$ is even"? \hfill \textbf{First}\, / \, \textbf{Second}
\end{itemize}

\subsection{Existence and Uniqueness}

\FRB{
{\bf Existence statements} assert that something exists. To prove them, construct a specific example and show it works!

{\bf Uniqueness statements} assert that only one thing satisfies a property. To prove uniqueness: Assume $a$ and $b$ both satisfy the property, then show that $a = b$.
}

\subsection{Example 1}

{\bf Proposition:} There exist integers $x, y, z$ such that $x^2 + y^2 = z^2$. 

\begin{proof}
Let $x = 3$, $y = 4$, and $z = 5$. These are all integers, and $3^2 + 4^2 = 9 + 16 = 25 = 5^2$. 
\end{proof}

\subsection{Example 2}

{\bf Proposition:} There exists a unique integer solution to $x^2 + 1 = 2x$. 

\begin{proof}
{\bf Existence:} Let $x = 1$. Then $1^2 + 1 = 1 + 1 = 2 = 2(1) = 2x$, so $x = 1$ is a solution.

{\bf Uniqueness:} Suppose $a$ is another integer solutions to the equation. We need to show that $a = 1$. Since $a$ is a solution, we know that $a^2 + 1 = 2a$.

Rearranging, we get that $(a-1)^2 = a^2 - 2a + 1 = 0$. Since the only number that squares to zero is $0$, we conclude that $a-1 = 0$, or equivalently that $a = 1$. \end{proof}

\subsection{Exercise}

For every integer $x$ there exists a unique integer $y$ such that $(x+1)^2 - x^2 = 2y - 1$. \vspace{5cm}

\section{Mathematical Induction}

\FRB{
{\bf Mathematical Induction} is a powerful technique for proving statements about all natural numbers.

If you have statements $S(1), S(2), S(3), \ldots$ and want to prove them all (i.e.\ prove $\forall n \in \mathbb{N}: S(n)$), you can:

\begin{itemize}
\item {\bf Base case:} Prove $S(1)$ is true
\item {\bf Inductive step:} Show that for any $k$, IF $S(k)$ is true THEN $S(k+1)$ is true
\end{itemize}

This is like dominoes: if the first falls and each domino knocks over the next, all will fall!
}

\subsection{Understanding Statements in Induction}

Before proving by induction, identify what $S(n)$ means for different values:

{\bf Problem:} If $n\geq 1$ then $6^n - 1$ is divisible by $5$.

$S(n)$: ``$6^n - 1$ is divisible by $5$'' \bigskip 

$S(1)$ is: \bigskip 

$S(6)$ is: \bigskip 

$S(k)$ is: \bigskip 

$S(k+1)$ is: \bigskip 

\bigskip

{\bf Problem:} $n^2 < 2^n$ for all $n \geq 5$.

$S(n)$ is: \bigskip

$S(5)$ is: \bigskip 

$S(k)$ is: \bigskip 

$S(k+1)$ is: \bigskip 

\newpage
\subsection{Writing Induction Proofs}

{\bf Template for induction proofs:}

\FR{
\begin{proof}
Let $S(n)$ be the statement ``[state what $S(n)$ is]''. We prove this by induction on $n$.

{\bf Base case:} We verify $S(1)$ [or whatever the starting value is]: [show the base case is true]

{\bf Inductive step:} Assume that $S(k)$ is true for some $k \geq 1$. That is, assume [state the inductive hypothesis].

We must show that $S(k+1)$ is true. That is, we must show [state what $S(k+1)$ says].

[Use the inductive hypothesis to prove $S(k+1)$]

{\bf Conclusion:} By mathematical induction, $S(n)$ is true for all $n \geq 1$. \qed
\end{proof}
}

\bigskip

{\bf Example:} The sum of the first $n$ odd positive integers is $n^2$. 

\FR{
\begin{proof}
Let $S(n)$ be the statement ``$\sum_{i = 1}^n (2i - 1) = n^2$''. We prove by induction on $n$.

{\bf Base case:} $S(1)$ asserts that $1 = 1^2$, which is true. 

{\bf Inductive step:} Assume that $S(k)$ is true for some $k \geq 1$. That is, assume 
$$\sum_{i = 1}^k (2i - 1) = k^2$$
This is the {\bf inductive hypothesis}.

We must show that $S(k+1)$ is true:
\begin{align*}
\sum_{i =1}^{k+1} (2i - 1) &= \left(\sum_{i =1}^k (2i - 1)\right) + (2(k+1) - 1) \\
&= k^2 + 2k + 1 \quad \text{(by inductive hypothesis)}\\
&= (k+1)^2
\end{align*}

{\bf Conclusion:} By mathematical induction, $S(n)$ is true for all $n \geq 1$. \qed
\end{proof}
}

{\bf Your Turn:} Prove the following by induction:

\begin{enumerate}
\item Let $n\geq 0$. Then $n^3 - n$ is divisible by $3$. 

\vfill

\newpage

\item Prove that if $n \geq 4$ then $n! \geq 2^n$.

\vfill


\item Prove that if $n$ is a positive integer then 
$$\sum_{i = 1}^n \frac{1}{i^2} \leq 2 - \frac{1}{n}.$$

\textit{Hint for the inductive step:} You'll need to show that 
$$2 - \frac{1}{k} + \frac{1}{(k+1)^2} \leq 2 - \frac{1}{k+1}$$

\vfill

\end{enumerate}

\newpage
\subsection{Strong Induction}

\FRB{
{\bf Strong Induction} is a variant where we assume ALL previous cases $S(1), S(2), \ldots, S(k)$ are true to prove $S(k+1)$.

Template:
\begin{itemize}
	\item Prove base case(s) (you might need more than one) 
	\item Show that IF $S(1), S(2), \ldots, S(k)$ are all true THEN $S(k+1)$ is true
\end{itemize}

This is like saying ``Show that IF dominoes $1, 2, 3, \ldots, k$ have fallen then domino $k+1$ will fall.''
}

\bigskip

{\bf Example:} You have 5-cent and 8-cent postage stamps. Prove that you can make any postage amount of 28 cents or more.

\begin{proof}
Let $S(n)$ be the statement ``You can make $n$ cents postage with 5 and 8 cent stamps.''

{\bf Base cases:} 
\begin{itemize}
\item $S(28)$: $28 = 4 \times 5 + 1 \times 8$ 
\item $S(29)$: $29 = 1 \times 5 + 3 \times 8$ 
\item $S(30)$: $30 = 6 \times 5$ 
\item $S(31)$: $31 = 3 \times 5 + 2 \times 8$ 
\item $S(32)$: $32 = 4 \times 8$ 
\end{itemize}

{\bf Inductive step:} Let $k \geq 32$ and assume $S(n)$ is true for all $28 \leq n \leq k$.

We must show $S(k+1)$ is true. Since $k + 1 \geq 33$, we have $(k+1) - 5 \geq 28$.

By the inductive hypothesis, $S((k+1) - 5)$ is true, so we can make $(k+1) - 5$ cents.

Adding one more 5-cent stamp gives us $k+1$ cents. Therefore $S(k+1)$ is true.

{\bf Conclusion:} By strong induction, we can make any postage amount of 28 cents or more.
\end{proof}

\bigskip

{\bf Your Turn:} 

You have a candy bar with $n$ squares of chocolate ($n\geq 1$) arranged in a rectangular grid. Prove that no matter how you break up the candy bar, one break at a time, it will take a total of $n-1$ breaks to break up the bar into individual squares.

\vfill

\newpage

{\bf Challenge Problems:}

Try to prove these statements using the techniques you've learned:

\begin{enumerate}

\item Ten children at a party collect a total of 74 pieces of candy from a pinata. Prove there is some child who received at least 8 pieces of candy. 

\vspace{3cm}

\item There exist 1000 positive integers in a row, none of which are prime.

\vspace{3cm}

\item {\bf Theorem:} Let $n > 1$ be an integer. Then $n$ is either prime or it can be written as a product of primes. (Use strong induction!)

\vfill

\item If $n$ is an integer that is not divisible by $2$ or $3$, then $n^2 - 1$ is divisible by $24$. 

\vspace{5cm}

\end{enumerate}

%
%\newpage
%
%\section{Common mistakes to avoid!}
%
%\subsection{Incorrect proofs}
%Sometimes we run into common mistakes. I'll give some proofs here, and your job is to tell me what is wrong with them. In some cases, I am `proving' incorrect statements, while in other cases, I am giving incorrect proofs of
%
%\begin{enumerate}
%\item If $n$ is an integer that is not divisible by $2$ or $3$, then $n^2 - 1$ is divisible by $24$. 
%\begin{proof}
%We will prove the contrapositive statement, that if $n^2 -1$ is not divisible by $24$, then $n$ is divisible by $2$ or $3$. 
%\end{proof}
%
%\end{enumerate}
%
%\subsection{Unclear proofs}
%Guidelines: 
%\begin{itemize}
%\item If you introduce a variable, be sure to say what set it lives in! 
%\item Use complete sentences when writing. 
%\item Guide the reader through what you are doing. E.g., if you want to do proof by contradiction, state at the beginning that you are doing proof by contradiction. 
%\end{itemize}
%
\end{document}
