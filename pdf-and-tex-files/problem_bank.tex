\documentclass[12pt]{exam}

\newcommand{\spc}[1]{{\color{teal}#1}}

% Symbols
\usepackage{amsmath, amsthm, amssymb}
\usepackage{mathtools}
\usepackage{wasysym}
\usepackage{mathrsfs}  
\usepackage{stmaryrd}
\usepackage{comment}
\usepackage{marginnote}

% Style
\usepackage[margin=1in]{geometry}
\usepackage{parskip}
\usepackage[shortlabels]{enumitem}

% Colored text
%\usepackage{color}
\usepackage{xcolor}

% Links
\usepackage[bookmarks,colorlinks,breaklinks]{hyperref}
% PDF hyperlinks, with coloured links
\hypersetup{linkcolor=red,citecolor=blue,
  filecolor=dullmagenta,urlcolor=blue}

% Bibliography and references
\usepackage[numbers]{natbib}
\usepackage[capitalize]{cleveref}

% Figures
\usepackage{graphicx}
\usepackage{tikz}
\usetikzlibrary{matrix, arrows}
\usetikzlibrary{cd}

\NeedsTeXFormat{LaTeX2e}
\ProvidesPackage{quiver}[2021/01/11 quiver]

% `tikz-cd` is necessary to draw commutative diagrams.
\RequirePackage{tikz-cd}
% `amssymb` is necessary for `\lrcorner` and `\ulcorner`.
\RequirePackage{amssymb}
% `calc` is necessary to draw curved arrows.
\usetikzlibrary{calc}
% `pathmorphing` is necessary to draw squiggly arrows.
\usetikzlibrary{decorations.pathmorphing}

% A TikZ style for curved arrows of a fixed height, due to AndréC.
\tikzset{curve/.style={settings={#1},to path={(\tikztostart)
    .. controls ($(\tikztostart)!\pv{pos}!(\tikztotarget)!\pv{height}!270:(\tikztotarget)$)
    and ($(\tikztostart)!1-\pv{pos}!(\tikztotarget)!\pv{height}!270:(\tikztotarget)$)
    .. (\tikztotarget)\tikztonodes}},
    settings/.code={\tikzset{quiver/.cd,#1}
        \def\pv##1{\pgfkeysvalueof{/tikz/quiver/##1}}},
    quiver/.cd,pos/.initial=0.35,height/.initial=0}
\usepackage[all]{xy}
   \SelectTips{cm}{10}

\usepackage{hyperref}
\usepackage{enumitem}
\usepackage{graphicx}

\makeatletter
\def\thm@space@setup{%
  \thm@preskip=0.5em\thm@postskip=\thm@preskip%
}
\makeatother
% Unnamed theorem without counter
\newtheorem*{theorem}{Theorem}
\newtheoremstyle{named}{}{}{\\itshape}{}{\bfseries}{.}{.5em}{\thmnote{#3's }#1}
% Named theorem without counter
\theoremstyle{named}
\newtheorem*{namedtheorem}{Theorem}
% Other theorem environments
\theoremstyle{plain}
\newtheorem{thm}{Theorem}[section]
\newtheorem{conj}[thm]{Conjecture}
\newtheorem{prop}[thm]{Proposition}
\newtheorem{lem}[thm]{Lemma}
\newtheorem{sublem}[thm]{Sub-Lemma}
\newtheorem{cor}[thm]{Corollary}
\theoremstyle{definition}
\newtheorem{defn}[thm]{Definition}
\newtheorem{ex}[thm]{Exercise}
\newtheorem{eg}[thm]{Example}
\newtheorem{assumption}[thm]{Assumption}
\newtheorem{convention}[thm]{Convention}
\newtheorem{fact}{Fact}
\crefformat{footnote}{#2\footnotemark[#1]#3}

%Hom
\newcommand{\Hom}{\mathrm{Hom}}

% Fonts and notation
\newcommand{\CC}{\mathbb{C}}
\newcommand{\QQ}{\mathbb{Q}}
\newcommand{\RR}{\mathbb{R}}
\newcommand{\ZZ}{\mathbb{Z}}

\newcommand{\mf}[1]{\mathfrak{#1}}
\newcommand{\mc}[1]{\mathcal{#1}}
\newcommand{\ms}[1]{\mathscr{#1}}
\newcommand{\mr}[1]{\mathrm{#1}}
\newcommand{\mbb}[1]{\mathbb{#1}}
\newcommand{\ol}[1]{\overline{#1}}
\newcommand{\ul}[1]{\underline{#1}}
\newcommand{\wt}[1]{\widetilde{#1}}
\newcommand{\wh}[1]{\widehat{#1}}

% Geometry
\newcommand{\Spa}{\mathrm{Spa}}
\newcommand{\Spec}{\mathrm{Spec}}
\newcommand{\Spf}{\mathrm{Spf}}
\newcommand{\Spd}{\mathrm{Spd}}

\newcommand{\an}{\mathrm{an}}

% sites
\newcommand{\proet}{\mr{pro\acute{e}t}}
\newcommand{\et}{\mr{\acute{e}t}}
\newcommand{\fet}{\mr{f\acute{e}t}}


%Algebraic groups
\newcommand{\GL}{\mathrm{GL}}
\newcommand{\SL}{\mathrm{SL}}
\newcommand{\SO}{\mathrm{SO}}
\newcommand{\OO}{\mathrm{O}}
\newcommand{\Sp}{\mathrm{Sp}}
\newcommand{\GSp}{\mathrm{GSp}}
\newcommand{\Spin}{\mathrm{Spin}}
\newcommand{\GSpin}{\mathrm{GSpin}}
\newcommand{\SU}{\mathrm{SU}}
\newcommand{\Stab}{\mathrm{Stab}}
\newcommand{\Rep}{\mathrm{Rep}}

% Miscellaneous
\newcommand{\Frob}{\mathrm{Frob}}
\DeclareMathOperator{\Aut}{Aut}
\newcommand{\colim}{\mathrm{colim}}
\renewcommand{\l}{\left}
\renewcommand{\r}{\right}

% p-adic geom
\newcommand{\crys}{\mathrm{crys}}
\newcommand{\Fil}{F}
\newcommand{\HT}{\mathrm{HT}}
\newcommand{\Gr}{\mathrm{Gr}}
\newcommand{\Fl}{\mathrm{Fl}}
\newcommand{\cris}{\mathrm{cris}}
\newcommand{\Isoc}{\mr{Isoc}} 
\newcommand{\dR}{\mathrm{dR}}
\newcommand{\Perf}{\mathrm{Perf}}
\newcommand{\Perfd}{\mathrm{Perfd}}
\newcommand{\FF}{\mathrm{FF}}


% Shimura varieties
\newcommand{\Sh}{\mathrm{Sh}}

% Coefficient objects for Weil cohomology theories
\newcommand{\Loc}{\mathrm{Loc}}
\newcommand{\MIC}{\mathrm{MIC}}

%Spacing
\newcommand{\+}{\, }


\begin{document}

\title{Problem bank}
\maketitle

\section{Simple computational questions}

\begin{question}
    \question Find all solutions to the congruence equation $5x^2 \equiv 1 \mod 11$ with $x \in \mathbb{Z}_{11} = \{0,1\ldots, 10\}$. 
\end{question}

\section{Exploratory worksheet}
\spc{I gave this in class on January 29th.}

For each of the statements below, determine whether they are true or false: give a proof if they are true or a counterexample if false. \emph{(Warning: Some statements might be difficult to prove/disprove, in which case you should guess whether you think the result is true or false and provide some computational evidence.)} 

\begin{questions}
\question Every prime number $p > 3$ is of the form $6k + 1$ or $6k-1$ for some integer $k$. 

\question Let $n > 2$ be an integer. 
\begin{parts}
    \part If $n \mid 2^{n-1} - 1$ then $n$ is prime. 
    \part If $n$ is prime, then $n\mid 2^{n-1} -1$. 
\end{parts}

\question 
\begin{parts}
    \part There are infinitely many primes of the form $3k + 1$. 
    \part There are infinitely many primes of the form $3k + 2$. 
\end{parts}

\question There are more primes of the form $3k + 2$ than of the form $3k +1$. More specifically, if 
    \[ \pi_1(n) = \#\{\text{primes $p \leq n$ of the for $3k+1$} \}, \]
    \[ \pi_2(n) = \#\{\text{primes $p \leq n$ of the for $3k+2$} \}, \]
then $\pi_1(n) \leq \pi_2(n)$ for any integer $n \geq 2$. 
\end{questions}

\section{Challenge problems}

\begin{questions}
   \question For every integer $n \geq 2$ there is a prime number between $n$ and $2n$. 

    \spc{This is Bertrands postulate.}

    \question $n$ is prime if and only if $a^n \equiv a \mod n$ for every integer $a$. 

    \spc{Fermat's little theorem, and the converse is false: Introduce Carmichael numbers. }

    \question There are infinitely many Carmichael numbers. 

    \spc{This is hard! Proven by Pomerance,...}

    \question Show that $n$ is a Carmichael number if and only if for every prime $p$ that divides $n$, $p-1$ also divides $n-1$.

    \spc{This is Korselt's criterion. I have no clue how easy it is to prove.}

    \question There are infinitely many Carmichael numbers in an arithmetic progression. 

    \spc{Also hard! Daniel Larsen's theorem. }
    
 
\end{questions}
\end{document}