\documentclass[12pt]{exam}


% Symbols
\usepackage{amsmath, amsthm, amssymb}
\usepackage{mathtools}
\usepackage{wasysym}
\usepackage{mathrsfs}  
\usepackage{stmaryrd}
\usepackage{comment}
\usepackage{marginnote}

% Style
\usepackage[margin=1in]{geometry}
\usepackage{parskip}
\usepackage[shortlabels]{enumitem}

% Colored text
%\usepackage{color}
\usepackage{xcolor}

% Links
\usepackage[bookmarks,colorlinks,breaklinks]{hyperref}
% PDF hyperlinks, with coloured links
\hypersetup{linkcolor=red,citecolor=blue,
  filecolor=dullmagenta,urlcolor=blue}

% Bibliography and references
\usepackage[numbers]{natbib}
\usepackage[capitalize]{cleveref}

% Figures
\usepackage{graphicx}
\usepackage{tikz}
\usetikzlibrary{matrix, arrows}
\usetikzlibrary{cd}

\NeedsTeXFormat{LaTeX2e}
\ProvidesPackage{quiver}[2021/01/11 quiver]

% `tikz-cd` is necessary to draw commutative diagrams.
\RequirePackage{tikz-cd}
% `amssymb` is necessary for `\lrcorner` and `\ulcorner`.
\RequirePackage{amssymb}
% `calc` is necessary to draw curved arrows.
\usetikzlibrary{calc}
% `pathmorphing` is necessary to draw squiggly arrows.
\usetikzlibrary{decorations.pathmorphing}

% A TikZ style for curved arrows of a fixed height, due to AndréC.
\tikzset{curve/.style={settings={#1},to path={(\tikztostart)
    .. controls ($(\tikztostart)!\pv{pos}!(\tikztotarget)!\pv{height}!270:(\tikztotarget)$)
    and ($(\tikztostart)!1-\pv{pos}!(\tikztotarget)!\pv{height}!270:(\tikztotarget)$)
    .. (\tikztotarget)\tikztonodes}},
    settings/.code={\tikzset{quiver/.cd,#1}
        \def\pv##1{\pgfkeysvalueof{/tikz/quiver/##1}}},
    quiver/.cd,pos/.initial=0.35,height/.initial=0}
\usepackage[all]{xy}
   \SelectTips{cm}{10}

\usepackage{hyperref}
\usepackage{enumitem}
\usepackage{graphicx}

\makeatletter
\def\thm@space@setup{%
  \thm@preskip=0.5em\thm@postskip=\thm@preskip%
}
\makeatother
% Unnamed theorem without counter
\newtheorem*{theorem}{Theorem}
\newtheoremstyle{named}{}{}{\\itshape}{}{\bfseries}{.}{.5em}{\thmnote{#3's }#1}
% Named theorem without counter
\theoremstyle{named}
\newtheorem*{namedtheorem}{Theorem}
% Other theorem environments
\theoremstyle{plain}
\newtheorem{thm}{Theorem}[section]
\newtheorem{conj}[thm]{Conjecture}
\newtheorem{prop}[thm]{Proposition}
\newtheorem{lem}[thm]{Lemma}
\newtheorem{sublem}[thm]{Sub-Lemma}
\newtheorem{cor}[thm]{Corollary}
\theoremstyle{definition}
\newtheorem{defn}[thm]{Definition}
\newtheorem{ex}[thm]{Exercise}
\newtheorem{eg}[thm]{Example}
\newtheorem{assumption}[thm]{Assumption}
\newtheorem{convention}[thm]{Convention}
\newtheorem{fact}{Fact}
\crefformat{footnote}{#2\footnotemark[#1]#3}

%Hom
\newcommand{\Hom}{\mathrm{Hom}}

% Fonts and notation
\newcommand{\CC}{\mathbb{C}}
\newcommand{\QQ}{\mathbb{Q}}
\newcommand{\RR}{\mathbb{R}}
\newcommand{\ZZ}{\mathbb{Z}}

\newcommand{\mf}[1]{\mathfrak{#1}}
\newcommand{\mc}[1]{\mathcal{#1}}
\newcommand{\ms}[1]{\mathscr{#1}}
\newcommand{\mr}[1]{\mathrm{#1}}
\newcommand{\mbb}[1]{\mathbb{#1}}
\newcommand{\ol}[1]{\overline{#1}}
\newcommand{\ul}[1]{\underline{#1}}
\newcommand{\wt}[1]{\widetilde{#1}}
\newcommand{\wh}[1]{\widehat{#1}}

% Geometry
\newcommand{\Spa}{\mathrm{Spa}}
\newcommand{\Spec}{\mathrm{Spec}}
\newcommand{\Spf}{\mathrm{Spf}}
\newcommand{\Spd}{\mathrm{Spd}}

\newcommand{\an}{\mathrm{an}}

% sites
\newcommand{\proet}{\mr{pro\acute{e}t}}
\newcommand{\et}{\mr{\acute{e}t}}
\newcommand{\fet}{\mr{f\acute{e}t}}


%Algebraic groups
\newcommand{\GL}{\mathrm{GL}}
\newcommand{\SL}{\mathrm{SL}}
\newcommand{\SO}{\mathrm{SO}}
\newcommand{\OO}{\mathrm{O}}
\newcommand{\Sp}{\mathrm{Sp}}
\newcommand{\GSp}{\mathrm{GSp}}
\newcommand{\Spin}{\mathrm{Spin}}
\newcommand{\GSpin}{\mathrm{GSpin}}
\newcommand{\SU}{\mathrm{SU}}
\newcommand{\Stab}{\mathrm{Stab}}
\newcommand{\Rep}{\mathrm{Rep}}

% Miscellaneous
\newcommand{\Frob}{\mathrm{Frob}}
\DeclareMathOperator{\Aut}{Aut}
\newcommand{\colim}{\mathrm{colim}}
\renewcommand{\l}{\left}
\renewcommand{\r}{\right}

% p-adic geom
\newcommand{\crys}{\mathrm{crys}}
\newcommand{\Fil}{F}
\newcommand{\HT}{\mathrm{HT}}
\newcommand{\Gr}{\mathrm{Gr}}
\newcommand{\Fl}{\mathrm{Fl}}
\newcommand{\cris}{\mathrm{cris}}
\newcommand{\Isoc}{\mr{Isoc}} 
\newcommand{\dR}{\mathrm{dR}}
\newcommand{\Perf}{\mathrm{Perf}}
\newcommand{\Perfd}{\mathrm{Perfd}}
\newcommand{\FF}{\mathrm{FF}}


% Shimura varieties
\newcommand{\Sh}{\mathrm{Sh}}

% Coefficient objects for Weil cohomology theories
\newcommand{\Loc}{\mathrm{Loc}}
\newcommand{\MIC}{\mathrm{MIC}}

%Spacing
\newcommand{\+}{\, }


\begin{document}

\title{Problem Set 1}
\date{Due: January 30th, 5pm}
\maketitle


\begin{questions}

\question \textbf{Primes in Arithmetic Progressions.}

Let $a,d \in \ZZ_{>0}$, and consider the arithmetic progression
\[
a,\, a+d,\, a+2d,\, a+3d,\, \dots
\]

Using computational experimentation, investigate for which pairs $(a,d)$ the arithmetic progression $a+nd$ contains infinitely many prime numbers. Try at least $6$ different pairs $(a,d)$. 

\emph{You are strongly encouraged to use SageMath.} In SageMath, the expression \texttt{i in Primes()} returns \texttt{True} if $i$ is prime and \texttt{False} otherwise.

Based on your experiments, state a conjecture characterizing all pairs $(a,d)$ for which the arithmetic progression contains infinitely many primes. Provide clear computational evidence supporting your conjecture (for example, tables, plots, or explicit data).

\emph{You are not expected to prove your conjecture.}

\question \textbf{The Well-Ordering Principle.}

Let $S \subseteq \RR$. The \emph{Well-Ordering Principle} for $S$ is the statement:

\begin{quote}
Every nonempty subset of $S$ has a least element (with respect to the usual order on $\RR$).
\end{quote}

 Give an explicit example of a nonempty subset $S \subseteq \RR$ for which the Well-Ordering Principle does not hold, and explain carefully why it fails.

\question \textbf{Divisibility Statements.}

Let $a,b,c$ be integers. Determine whether each statement is true or false. If true, give a proof. If false, give a counterexample.

\begin{parts}
\part If $a \mid c$ and $b \mid c$, then $a+b \mid c$.
\part If $a \mid b$ and $a \mid c$, then $a \mid (b+c)$.
\end{parts}

\break 

\question \textbf{Linear Diophantine Equations.}

Find \emph{all} integer solutions to the equation
\[
252x + 198y = 18.
\]

\question \textbf{Congruences.}

Let $a,b \in \ZZ$ and let $n \in \ZZ_{>0}$. We say that $a \equiv b \pmod n$ or that $a$ is \emph{congruent to $b$ modulo $n$} if $n \mid (b-a)$. Using this definition, prove the following properties. 

\begin{parts}

\part Find $a \in \{0,1, \ldots, 11\}$ such that $7^5 + a \equiv 2 \pmod{12}$.

\part Using the definition, prove the following properties:
\begin{enumerate}[(i)]
\item $a \equiv a \pmod{n}$
\item If $a \equiv b \pmod{n}$, then $b \equiv a \pmod{n}$
\item If $a \equiv b \pmod{n}$ and $b \equiv c \pmod{n}$, then $a \equiv c \pmod{n}$
\item If $a \equiv b \pmod{n}$, then $a+c \equiv b+c \pmod{n}$ for all $c \in \ZZ$.
\end{enumerate}
\end{parts}

\question \textbf{The Euclidean Algorithm.}

Let $a$ and $b$ be positive integers with $a>b$, and let
\[
b = r_0, r_1, r_2, \dots
\]
be the successive remainders in the Euclidean algorithm applied to $a$ and $b$.

\begin{parts}
\part Show that after every two steps, the remainder is reduced by at least one half. In other words, verify that
\[
r_{i+2} < \frac{1}{2} r_i
\]
for every $i=0,1,2,\dots$.

\part Conclude that the Euclidean algorithm terminates in at most $2\log_2(b)$ steps, where $\log_2$ denotes the logarithm to the base $2$.

\part Deduce that the number of steps is at most seven times the number of decimal digits of $b$.

\emph{Hint: What is the value of $\log_2(10)$?}
\end{parts}

\question \textbf{The $3n+1$ Algorithm.}

The \emph{$3n+1$ algorithm} works as follows. Start with any positive integer $n$.
\begin{itemize}
\item If $n$ is even, divide it by $2$.
\item If $n$ is odd, replace it with $3n+1$.
\end{itemize}
Repeat this process indefinitely.

For example, starting with $5$ we obtain
\[
5,16,8,4,2,1,4,2,1,4,2,1,\dots
\]
and starting with $7$ we obtain
\[
7,22,11,34,17,52,26,13,40,20,10,5,16,8,4,2,1,4,2,1,\dots
\]

Notice that if the algorithm ever reaches $1$, the sequence repeats $4,2,1$ forever.

In general, one of the following two possibilities occurs:
\begin{enumerate}
\item The sequence eventually repeats a number that appeared earlier. In this case, the block of numbers between the two occurrences repeats indefinitely. We say that the algorithm \emph{terminates} at the last nonrepeated value, and the number of distinct entries is called the \emph{length} of the algorithm.
\item The sequence never repeats any value, in which case we say that the algorithm does not terminate.
\end{enumerate}

\begin{parts}
\part Verify that the algorithm terminates at $1$ for the starting values $5$ and $7$, and compute the length of the algorithm in each case.
\part Write a short program or use computational software to test the algorithm for many starting values. What do you observe?
\part State a conjecture about whether the $3n+1$ algorithm terminates for all positive integers.
\end{parts}

\end{questions}

\end{document}
